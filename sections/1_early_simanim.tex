%%%%% KADESH
\siman{
קַדֵּשׁ
}

\instruction{
הברכות הללו נאמרים אל כוס הראשון. 
}{These blessings are said over the first cup.}

\begin{framed}
\instruction{
בשבת מתחילין:
}{On Shabos, begin:}

\pasuk{
וַיְהִי עֶרֶב וַיְהִי בֹקֶר יוֹם הַשִּׁשִּׁי. וַיְכֻלּוּ הַשָׁמַיִם וְהָאָרֶץ וְכָל צְבָאָם. וַיְכַל אֱלֹהִים בַּיוֹם הַשְּׁבִיעִי מְלַאכְתּוֹ אֲשֶׁר עָשָׂה וַיִּשְׁבֹּת בַּיוֹם הַשְּׁבִיעִי מִכָּל מְלַאכְתּוֹ אֲשֶׁר עָשָׂה. וַיְבָרֶךְ אֱלֹהִים אֶת יוֹם הַשְּׁבִיעִי וַיְקַדֵּשׁ אוֹתוֹ כִּי בוֹ שָׁבַת מִכָּל מְלַאכְתּוֹ אֲשֶׁר בָּרָא אֱלֹהִים לַעֲשׂוֹת.
}
\end{framed}

\begin{center}
סַבְרִי מָרָנָן וְרַבָּנָן וְרַבּוֹתַי

בָּרוּךְ אַתָּה יי אֱלֹהֵינוּ מֶלֶךְ הָעוֹלָם בּוֹרֵא פְּרִי הַגָפֶן.
\end{center}

\instruction{
בשבת אומרים הדברים המוספים:
}{On Shabos say the extra words:}

בָּרוּךְ אַתָּה יי אֱלֹהֵינוּ מֶלֶךְ הָעוֹלָם, אֲשֶׁר בָּחַר בָּנוּ מִכָּל עָם וְרוֹמְמָנוּ מִכָּל לָשׁוֹן וְקִדְּשָׁנוּ בְּמִצְוֹתָיו. וַתִּתֶּן לָנוּ יי אֱלֹהֵינוּ בְּאַהֲבָה (שַׁבָּתוֹת לִמְנוּחָה וּ)מוֹעֲדִים לְשִׂמְחָה, חַגִּים וּזְמַנִּים לְשָׂשׂוֹן, אֶת יוֹם (הַשַּׁבָּת הַזֶה וְאֶת יוֹם) חַג הַמַּצּוֹת הַזֶּה, זְמַן חֵרוּתֵנוּ (בְּאַהֲבָה), מִקְרָא קֹדֶשׁ, זֵכֶר לִיצִיאַת מִצְרָיִם. כִּי בָנוּ בָחַרְתָּ וְאוֹתָנוּ קִדַּשְׁתָּ מִכָּל הָעַמִּים, (וְשַׁבָּת) וּמוֹעֲדֵי קָדְשֶךָ (בְּאַהֲבָה וּבְרָצוֹן,) בְּשִׂמְחָה וּבְשָׂשׂוֹן הִנְחַלְתָּנוּ. בָּרוּךְ אַתָּה יי, מְקַדֵּשׁ (הַשַּׁבָּת וְ)יִשְׂרָאֵל וְהַזְּמַנִּים.

\begin{framed}
\instruction{
במוצאי שבת מוסיפין
}{On Motsei Shabos, add:}

בָּרוּךְ אַתָּה יי אֱלֹהֵינוּ מֶלֶךְ הָעוֹלָם, בּוֹרֵא מְאוֹרֵי הָאֵשׁ.

בָּרוּךְ אַתָּה יי אֱלֹהֵינוּ מֶלֶךְ הָעוֹלָם הַמַּבְדִּיל בֵּין קֹדֶשׁ לְחֹל, בֵּין אוֹר לְחשֶׁךְ, בֵּין יִשְׂרָאֵל לָעַמִּים, בֵּין יוֹם הַשְּׁבִיעִי לְשֵׁשֶׁת יְמֵי הַמַּעֲשֶׂה. בֵּין קְדֻשַּׁת שַׁבָּת לִקְדֻשַּׁת יוֹם טוֹב הִבְדַּלְתָּ, וְאֶת יוֹם הַשְּׁבִיעִי מִשֵּׁשֶׁת יְמֵי הַמַּעֲשֶׂה קִדַּשְׁתָּ. הִבְדַּלְתָּ וְקִדַּשְׁתָּ אֶת עַמְּךָ יִשְׂרָאֵל בִּקְדֻשָּׁתֶךָ. בָּרוּךְ אַתָּה יי הַמַּבְדִיל בֵּין קֹדֶשׁ לְקֹדֶשׁ.
\end{framed}

בָּרוּךְ אַתָּה יי אֱלֹהֵינוּ מֶלֶךְ הָעוֹלָם, שֶׁהֶחֱיָנוּ וְקִיְּמָנוּ וְהִגִּיעָנוּ לַזְּמַן הַזֶה.

\begin{center} \instruction{
שותה רב כוס היין בהסבה
}{Drink most of the first cup, reclining.}
\end{center}

%%%%% URHATS
\siman{
וּרְחַץ
}

\instruction{
נוטלין את הידים אבל אין מברכין.
}{Wash your hands, but do not make a blessing.}

%%%%% KARPAS
\siman{
כַּרְפַּס
}

\instruction{
טובלין כרפס במי מלח ומברכין עליו:
}{Dunk a vegetable in salt water and bless it:}

בָּרוּךְ אַתָּה יי אֱלֹהֵינוּ מֶלֶךְ הָעוֹלָם, בּוֹרֵא פְּרִי הָאֲדָמָה.

%%%%% YAHATS
\siman{
יַחַץ
}

\instruction{
בעל הבית יבצע את המצה האמצעית לשתים ומצפין את החצי הגדול לאפיקומן
}{The head of the house breaks the middle matso into two pieces, and sets aside the larger of the pieces to be the afikomen.}